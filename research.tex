\chapter{Approach}
% TODO: remember to justify the choices made
% also compare different possibilities

In this chapter, the work towards modeling a moving human is described. Starting out, there was much to be figured out and tried. Many questions were not only unanswered but still unasked. The target was already defined: a rigged 3D mesh of the user. The tools available were a Microsoft Kinect for Xbox 360 and a desktop computer.

\section{Obtaining point cloud data}

The first step towards achieving anything with a Kinect is, obviously, to read its sensor data to the computer. This is not trivially easy considering that Kinect was originally only meant to be an Xbox 360 accessory. Each of the three drivers mentioned in section \ref{literature.drivers} were installed and tested by writing trivial software using them.

Skeleton tracking using the Microsoft SDK was tested by compiling a sample application and making modifications to it. The API seems to be good and usable, and the skeleton tracking methods work quite well. Fast movement and occlusion caused problems such as the skeleton jumping into weird poses for a short time.

OpenNI was tested using the SimpleOpenNI \citep{simpleopenni} wrapper for Processing \citep{processing}. The included examples were plentiful and diverse, and allowed for quickly trying out own ideas (probably in part due to familiarity with the Processing environment).

Of the three drivers, libfreenect was the easiest to start working with. By installing the OpenKinect plugin for Processing \citep{shiffman2010} \citep{processing}, the Kinect was up and running in minutes. Development was easy given the bundled examples, which include creating point clouds as shown by \citet{fisher2010}. Apparently the functionality is still quite low-level, and thus not very suitable for trying to capture human body details.

Limited resources were allocated for this research project, so practically skeleton tracking was a requirement for the sensor software. The choice was then between completely proprietary, single-platform Microsoft SDK with a highly restrictive license and the partially proprietary, multi-platform OpenNI/NITE combination with unclear licensing. In preliminary testing, differences in the accuracy of skeleton tracking were minor.

OpenNI was finally chosen to be used for data acquisition in this work. The Microsoft SDK would probably have been similarly useful, but was not chosen because OpenNI allows for multiple platforms and is more open.

\section{Experiments with point clouds}

\begin{figure}
    \centering
    \includegraphics[width=\textwidth]{pcd-plain.png}
    \caption{A point cloud constructed from a single frame of Kinect data. The cloud is captured and saved in the .pcf format (Point Cloud Format) that PCL uses. The viewer used is part of PCL.}
    \label{fig:pcd-plain}
\end{figure}


\section{Body part segmentation}

Before any attempts at human body reconstruction can be made, the body needs to be detected from the RGB-D image. % mention 'segmentation'

The NITE middleware for OpenNI \citep{NITE}
includes functions for person detection. Moreover, NITE has the capability to generate a skeleton representation of human users.

As the skeleton representation is available, we used it to aid in reconstructing the body model.

Using the skeleton data, we segment the point cloud according to body parts.

\fixme{The body parts used were chosen according to what the NITE skeleton allowed.} Thus the following segments were used:

\fixme{TODO: what do we really have? what are the right English terms?
legs, feet, forearms, shoulders, head, torso, hip}

\begin{figure}
    \centering
    \includegraphics[width=\textwidth]{pcd-segmented.png}
    \caption{The same point cloud as in \ref{fig:pcd-plain}, with user segmented from the background, and further segmented into individual body parts.}
    \label{fig:pcd-segmented}
\end{figure}

\section{Point cloud alignment}

The naïve approach to modeling uses data only from a single frame. However, this is unsatisfactory as in practice only less than a half (e.g. the front part) of a human can be seen at once. Another consideration is that the data tends to be noisy and inaccurate. Accumulating data over time makes it possible to remove some of the noise and improve accuracy.

To allow for combining data from multiple frames, the observations (point clouds) need to be aligned one way or another. Methods introduced in \autoref{literature.alignment} were available, each with their own trade-offs.

\fixme{write more about this when the best approach is actually chosen}

\section{Mesh generation}

\fixme{Alternative approaches, including the following}

\subsection{Kinfu and automatic rigging}

\begin{figure}
    \centering
    \fixme{TODO: picture of the rigged mesh}
    \label{fig:hannu-mesh}
\end{figure}

\citet{charpentier2011accurate} shows how a rigged mesh model of the user can be made by manually aligning shots from different directions and then using an automatic rigging algorithm. \citeauthor{charpentier2011accurate} claims this approach would become very easy, if the KinectFusion system were available.

As the Kinfu implementation was available, this approach was tested in practice. The amount of manual work required turned out to still be surprisingly large.

The software needed for this approach are the KinectFusion system for scanning the mesh, a mesh editor for making necessary changes in the mesh and an auto-rigging software. We used MeshLab and Pinocchio for the latter two purposes, as \citet{charpentier2011accurate} had also successfully employed them. At the time of writing, Kinfu and Pinocchio needed to be compiled manually. Both have command-line interfaces that take a little getting used to; similarly, MeshLab's GUI (Graphical User Interface) is slightly confusing for a beginner.

The actual scanning requires two persons: one to be scanned and one to scan. The scanned subject needs to remain still while the other person carefully and slowly goes around them, slowly moving the Kinect so that as much of the body as possible can be seen. After this is done, the user presses a key to extract a mesh from the TSDF. This operation can only be completed if the GPU has at least 1.5 gigabytes of memory; otherwise, it's possible to change Kinfu source code to lower the TSDF resolution and recompile. The extracted mesh is then saved to disk by pressing another key.

The extracted mesh contains the subject and the surrounding area. The subject needs to be cropped for the automatic rigging. This requires manual mesh editing. In our test run, the full mesh was 140MB in size and comprised some 5 million vertices and 1.5 million faces. Editing this mesh made MeshLab a bit sluggish on a high-end desktop computer\footnote{Intel Core i7-3770, 16GB RAM, SSD and an Nvidia GeForce GTX670.}. For comparison, the same mesh was imported in Blender, after which Blender became stuck for about a minute and then showed the mesh while remaining quite unusable.

After the mesh is cropped and only the human part remains, the actual editing can begin. Pinocchio needs a mesh that is closed and connected, and the one produced by Kinfu is very probably not. One possible approach to this problem is to uniformly sample a subset of the vertices and then use a mesh generation algorithm on those vertices. The downside is that some accuracy is lost, but on the other hand the regions with holes are nicely filled. Whatever approach is taken, it might take some trial and error to choose good algorithms and parameters. The computation is not instant, but not necessarily slow either, again depending on the exact methods used.

With this manual editing of the mesh finished, it is time to complete the automatic rigging using Pinocchio. In our tests, the first two tries failed because Pinocchio expects the mesh to be oriented in a certain way. This is not difficult to fix, but manual intervention is still needed at this stage.

Finally, Pinocchio successfully rigged the mesh and a walking animation of the user was seen on the screen---quite astonishing! Minor details such as slightly wrong placement of the shoulder joints made the animation a little unnatural, but the result still causes amazement. The mesh also has rough surface details, caused by drift in the camera position and lack of loop closure in Kinfu. This is shown in figure~\ref{fig:hannu-mesh}.

All in all, the amount of manual work required to create a rigged mesh from scratch was about an hour with some earlier experience of the tools. The process could possibly be automated, but implementing such an automatic system is far from trivial. The scanning phase would still necessarily need another person to move the Kinect around, and the time requirements would probably be around a minute for scanning and five minutes for processing. The method is thus feasible, but not as easy as \citet{charpentier2011accurate} suggests.

\subsection{Voxel grid}

One possible approach is to use a voxel grid similar to the one KinectFusion \autocites{newcombe2011kinectfusion}{izadi2011kinectfusion} uses. This allows generating meshes representing arbitrary shapes.

To evaluate the approach, a prototype was built on top of Kinfu\footnote{Kinfu is an open source implementation of KinectFusion \citep{newcombe2011kinectfusion}. It is included in the Point Cloud Library \citep{PCL}.}. Since the point clouds were already segmented by body part, it was possible to create recordings that only include a single body part. These recordings could then be played back and used as input for Kinfu.

The working hypothesis was that each body part could be treated as a static object, and that Kinfu should do quite well at modeling them. \fixme{Notably there's little difference between the camera moving} (as is the case in KinectFusion) and the object moving. If the background is filtered out, the result is similar. This made for the case that running the KinectFusion algorithm in parallel to each body part could give reasonably good results.

In practice, Kinfu doesn't work well with isolated limb data. This has multiple causes. \fixme{1) For full-body scanning the whole user must obviously fit in the picture. At the VGA resolution that Kinect uses, this leaves few pixels per limb. 2) The depth resolution of Kinect is about 1--2 centimeters at a distance suitable for full-body scanning. 3) ICP only uses surface features for alignment, not color data. As the individual body parts tend to be quite smooth, this is a problem.}

% TODO: split into three different images
\begin{figure}
    \centering
    \includegraphics[width=\textwidth]{limbwise-head.png}
    \caption{Kinfu running on a view with only the head.}
    \label{fig:limbwise-head}
\end{figure}

% TODO: split into three different images
\begin{figure}
    \centering
    \includegraphics[width=\textwidth]{limbwise-arm.png}
    \caption{Kinfu running on a view with only the left arm.}
    \label{fig:limbwise-arm}
\end{figure}

\subsection{Parametric surfaces}

Another approach is to use the information readily available about the body parts being modeled. Instead of allowing arbitrary shapes, the space of possible shapes can be limited to what the body parts tend to look like.

As the very first prototype, a simple cylinder approximation was used. A cylinder was fitted to the point cloud of a body part, and colored according to its average color. This was good for testing the segmentation and getting a general idea of how accurate the skeleton is. Some corner cases were also found using this prototype. \fixme{elaborate...} \fixme{TODO: screenshot}

Obviously, a more human-like model was needed. The "cylinder man" could be usable as an avatar, and is certainly recognizable as human-like. But certainly its shape was nowhere near a real human body. Different simple shapes such as ellipsoids were considered. This would still have left the problem of connecting the body parts. Some kind of a meta-blob solution might have been feasible, but in the end this approach started to seem quite cumbersome. And still the accuracy would have left much to be hoped for.

The best parametric model for the purpose would then be something that already assumes the generic human shape, while allowing \fixme{exact} modifications to single body parts. The SCAPE model \citep{anguelov2005scape} used by \citet{weiss2011home} fits the description, but is not suitable for real-time evaluation. A more suitable approach is taken by MakeHuman \citep{makehuman}, which uses a base mesh and has defined parameters that can be used to reshape different parts of the mesh.

MakeHuman is still in development and at the time of writing was undergoing some large changes. It is not designed to be used as a library, either. Interfacing with other software thus seemed nontrivial. After some time tinkering with the MakeHuman implementation, it was decided that the data is important while the software can be replaced. The most important features were not very difficult to implement.

\fixme{Where should the implementation details of MakeHuman be described?}

\begin{figure}
    \centering
    \includegraphics[width=\textwidth]{makehuman-measurement.png}
    \caption{MakeHuman with the Measurement plugin active. Waist circumference is selected. The white line shows where the circumference is measured.}
    \label{fig:makehuman-measurement}
\end{figure}

\begin{figure}
    \centering
    \includegraphics[width=\textwidth]{mhtargets.png}
    \caption{The reimplementation of the MakeHuman target system, showing the unmodified base mesh.}
    \label{fig:mhtargets}
\end{figure}

\begin{figure}
    \centering
    \includegraphics[width=\textwidth]{mhtargets-fat.png}
    \caption{The model in \ref{fig:mhtargets} with changed front chest, bust, underbust, waist and hip measurements.}
    \label{fig:mhtargets-fat}
\end{figure}

\section{Texturing}

\fixme{Write something here once I've done something...}


