% thesis using aalto-thesis.sty
% http://cse.aalto.fi/studies/instructions/tips-for-masters-thesis-workers/
\documentclass[12pt,a4paper,oneside]{report}

% try to maintain compatibility both with pdftex and xetex
\usepackage{ifxetex}
\ifxetex
  \usepackage{fontspec}
\else
  \usepackage[utf8]{inputenc}
  \usepackage[OT1]{fontenc}
\fi

\usepackage[english,finnish]{babel}
\usepackage{csquotes}

% Bibliography stuff
\usepackage[
%    backend=biber,
    style=authoryear,   % Author (year) / (Author, year) citations
    natbib=true,        % enable \citep and \citet
    maxbibnames=20,     % print all authors in bibliography
    dashed=false,       % print author even if it's the same as previous
    firstinits,         % replace first names with just the initial
%    backref=true, % for checking usage of sources
        ]{biblatex}
\bibliography{sources.bib}

% insert a 'Bibliography' item in TOC
\usepackage[nottoc]{tocbibind}

\usepackage{graphicx}
\graphicspath{{./images/}}

\usepackage{algorithmicx}
\usepackage{algorithm}
\usepackage{algpseudocode}

\usepackage{longtable}

\usepackage{amsfonts}

% The aalto-thesis package provides typesetting instructions for the
% standard master's thesis parts (abstracts, front page, and so on)
% Load this package second-to-last, just before the hyperref package.
% Options that you can use: 
%   mydraft - renders the thesis in draft mode. 
%             Do not use for the final version. 
%   doublenumbering - [optional] number the first pages of the thesis
%                     with roman numerals (i, ii, iii, ...); and start
%                     arabic numbering (1, 2, 3, ...) only on the 
%                     first page of the first chapter
%   twoinstructors  - changes the title of instructors to plural form
%   twosupervisors  - changes the title of supervisors to plural form
\usepackage[mydraft,doublenumbering]{aalto-thesis}

% Hyperref
% ------------------------------------------------------------------
% Hyperref creates links from URLs, for references, and creates a
% TOC in the PDF file.
% This package must be the last one you include, because it has
% compatibility issues with many other packages and it fixes
% those issues when it is loaded.   
\RequirePackage{hyperref}
% Setup hyperref so that links are clickable but do not look 
% different
\hypersetup{colorlinks=false,raiselinks=false,breaklinks=true}
\hypersetup{pdfborder={0 0 0}}
\hypersetup{bookmarksnumbered=true}
% The following line suggests the PDF reader that it should show the 
% first level of bookmarks opened in the hierarchical bookmark view. 
\hypersetup{bookmarksopen=true,bookmarksopenlevel=1}
% Hyperref can also set up the PDF metadata fields. These are
% set a bit later on, after the thesis setup.   

% Thesis setup
% ==================================================================
% Change these to fit your own thesis.
% \COMMAND always refers to the English version;
% \FCOMMAND refers to the Finnish version; and
% \SCOMMAND refers to the Swedish version.
% You may comment/remove those language variants that you do not use
% (but then you must not include the abstracts for that language)
% ------------------------------------------------------------------
\newcommand{\TITLE}{Real-time Human Body Reconstruction Using a Depth Camera}
\newcommand{\SUBTITLE}{}
\newcommand{\DATE}{2012-xx-xx}

\newcommand{\SUPERVISOR}{Professor Tapio Takala}
\newcommand{\INSTRUCTOR}{Tuukka Takala M.Sc. (Tech.)}


% Other stuff
% ------------------------------------------------------------------
\newcommand{\PROFESSORSHIP}{Media Technology}
% Professorship code is the same in all languages
\newcommand{\PROFCODE}{T-111}
\newcommand{\KEYWORDS}{3D reconstruction, avatar creation, body scanning}
\newcommand{\LANGUAGE}{English}

% Author is the same for all languages
\newcommand{\AUTHOR}{Hannu Hartikainen}


% Finnish translations (TODO)
\newcommand{\FTITLE}{TODO}
\newcommand{\FSUBTITLE}{}
\newcommand{\FSUPERVISOR}{Professori Tapio Takala}
\newcommand{\FINSTRUCTOR}{DI Tuukka Takala}
\newcommand{\FPROFESSORSHIP}{Mediatekniikka}
\newcommand{\FKEYWORDS}{TODO}
\newcommand{\FLANGUAGE}{englanti}
\newcommand{\FDATE}{\DATE}


% Currently the English versions are used for the PDF file metadata
% Set the PDF title
\hypersetup{pdftitle={\TITLE\ \SUBTITLE}}
% Set the PDF author
\hypersetup{pdfauthor={\AUTHOR}}
% Set the PDF keywords
\hypersetup{pdfkeywords={\KEYWORDS}}
% Set the PDF subject
\hypersetup{pdfsubject={Master's Thesis}}


% Custom commands
% ------------------------------------------------------------------
\newcommand{\term}[1]{\textit{#1}}
\newcommand{\bs}[1]{#1}
\newcommand{\newtopic}{\vspace{0.5cm}}

\newcommand{\code}[1]{\begin{verse}#1\end{verse}}


% Layout settings
% ------------------------------------------------------------------

% Use this to control how much space there is between each line of text.
% 1 is normal (no extra space), 1.3 is about one-half more space, and
% 1.6 is about double line spacing.  
\linespread{1.3}


% The preamble ends here, and the document begins. 
% Place all formatting commands and such before this line.
% ------------------------------------------------------------------
\begin{document}
% This command adds a PDF bookmark to the cover page. You may leave
% it out if you don't like it...
%\pdfbookmark[0]{Cover page}{bookmark.0.cover}
% This command is defined in aalto-thesis.sty. It controls the page 
% numbering based on whether the doublenumbering option is specified
\startcoverpage

% Cover page
% ------------------------------------------------------------------
% Options: finnish, english, and swedish
% These control in which language the cover-page information is shown
\coverpage{english}


% Abstracts
% ------------------------------------------------------------------
% Include an abstract in the language that the thesis is written in,
% and if your native language is Finnish or Swedish, one in that language.

% Abstract in English
% ------------------------------------------------------------------
%\pdfbookmark[0]{Abstract}{bookmark.0.abstract}
\thesisabstract{english}{
Depictions of humans have been made for thousands of years. As technology has progressed, new ways of replicating the human form visually have become available. Currently, virtual 3D characters can be created, but making them requires a lot of skill and effort.

Depth cameras have become cheap and are growing commonplace. This makes 3D scanning at home possible. Software for scanning static objects is already available. However, reconstructing a moving human is difficult. Previously suggested methods are slow and no implementation has been made public.

We research possible methods for automatically generating a 3D model that looks similar to the user and can be animated. We do this using a consumer depth camera that is used to track a freely moving user. We also attempt to produce an implementation that works in real-time on a high-end home computer. 

The accomplishments of this thesis are threefold. First, we review related work and existing methods and identify relevant areas of research. Secondly, different feasible approaches are planned and tried, and both their benefits and disadvantages are discussed. We show what different choices can be made in regard to building a reconstruction system, and what problems must be solved. Finally, we create some useful prototype implementations, while we fall short of a complete automatic reconstruction system.
}

\thesisabstract{finnish}{
    \fixme{TODO: suomenna Abstract, kunhan valmis}
}


% Acknowledgements
% ------------------------------------------------------------------
\selectlanguage{english}
%\pdfbookmark[0]{Acknowledgements}{bookmark.0.acknowledgements}
\chapter*{Acknowledgements}

I wish to thank the Department of Media Technology at Aalto University School of Science, which employed me to conduct this research and write a thesis. I'm especially grateful to my supervisor Tapio Takala and my instructor Tuukka Takala. I was also happy to work with Klaus Förger, Mikael Matveinen, Meeri Mäkäräinen and Roberto Pugliese in TUAS room 2550.

During this work, the greatest support has come from my wife Katri---helping me relax and get my mind off work in my free time. I love you.

Others I'd like to express my gratitude include my mother and father and extended family for bringing me this far; the people at Ristin kilta for being friends to me; the Källi brothers Sakari Pekka Paavali, Juho Arvi Samuel and Jussi Chafa Pappa for all the fun that always kept me sane; Johannes Haataja for helping me understand academia and laughing with me, not at me; and all my friends and all the people who have supported me in the past 26 years. I'm sorry that I can't mention every one of you.

Finally, I'm in an inexpressible and incomprehensible debt to my savior Jesus Christ, who has died for my sins and always loved me no matter how much sorrow I have caused Him.

\vskip 10mm

\noindent Espoo, \DATE
\vskip 5mm
\noindent\AUTHOR


% Acronyms
% ------------------------------------------------------------------
% Use \cleardoublepage so that IF two-sided printing is used 
% (which is not often for masters theses), then the pages will still
% start correctly on the right-hand side.
%\cleardoublepage
\addcontentsline{toc}{chapter}{Abbreviations and Acronyms}
\chapter*{Abbreviations and Acronyms}

% The longtable environment should break the table properly to multiple pages, 
% if needed

\noindent
\begin{longtable}{@{}p{0.25\textwidth}p{0.7\textwidth}@{}}
        EM & Expectation-Maximization \\
    EM-ICP & an EM-based rigid registration method comparable to ICP \\
     FOVIS & Fast Odometry from VISion \\
       ICP & Iterative Closest Point \\
       PCL & Point Cloud Library \\
     RGB-D & red, green, blue, depth; a combination of image and range data \\
      SLAM & Simultaneous localization and mapping \\
   TPS-RPM & \\
      TSDF & Truncated Signed Distance Function \\
\end{longtable}



% Table of contents
% ------------------------------------------------------------------
\cleardoublepage
% This command adds a PDF bookmark that links to the contents.
% You can use \addcontentsline{} as well, but that also adds contents
% entry to the table of contents, which is kind of redundant.
% The text ``Contents'' is shown in the PDF bookmark. 
\pdfbookmark[0]{Contents}{bookmark.0.contents}
\tableofcontents

% The following label is used for counting the prelude pages
\label{pages-prelude}
\cleardoublepage


%%%%%%%%%%%%%%%%% The main content starts here %%%%%%%%%%%%%%%%%%%%%
% ------------------------------------------------------------------
% This command is defined in aalto-thesis.sty. It controls the page 
% numbering based on whether the doublenumbering option is specified
\startfirstchapter

% Add headings to pages (the chapter title is shown)
\pagestyle{headings}

% The contents of the thesis are separated to their own files.
% Edit the content in these files, rename them as necessary.
% ------------------------------------------------------------------

\chapter{Introduction}

\fixme{TODO: write introduction}

\chapter{Related work}

\fixme{Refer to relevant sources a topic at a time}

\section{Point cloud alignment}
\citep{huang2011visual}
\citep{rusinkiewicz2001efficient}
\citep{tamaki2010softassign}
\citep{granger2006multi}
\citep{tykkala2011direct}

\section{Human body models}
\citep{anguelov2005scape}
\citep{baek2012parametric}

\section{3D reconstruction}
(generic)
\citep{fabio2003point}

(kinect fusion)
\citep{izadi2011kinectfusion}
\citep{newcombe2011kinectfusion}
\citep{Whelan12rssw}

(human mesh)
\citep{weiss2011home}
\citep{schneider2010fitting}
\citep{ahmed2005automatic}
\citep{tongscanning}
\citep{charpentier2011accurate}
\citep{hirshbergc2011evaluating}

(human pose)
\citep{baak2011data}

\chapter{Approach}
% TODO: remember to justify the choices made
% also compare different possibilities

\section{Obtaining point cloud data}

% TODO: explain RGB-D
% TODO: avoid using an abbreviation altogether?

The OpenNI library \citep{OpenNI}
was used for data acquisition. The actual sensor used was Microsoft Kinect.

\section{Body part segmentation}

Before any attempts at human body reconstruction can be made, the body needs to be detected from the RGB-D image. % mention 'segmentation'

The NITE middleware for OpenNI \citep{NITE}
includes functions for person detection. Moreover, NITE has the capability to generate a skeleton representation of human users.

As the skeleton representation is available, we used it to aid in reconstructing the body model.

Using the skeleton data, we segment the point cloud according to body parts.

\fixme{The body parts used were chosen according to what the NITE skeleton allowed.} Thus the following segments were used:

\fixme{TODO: what do we really have? what are the right English terms?
legs, feet, forearms, shoulders, head, torso, hip}

\section{Point cloud alignment}

The naïve approach to modeling uses data only from a single frame. However, this is unsatisfactory as in practice only less than a half (e.g. the front part) of a human can be seen at once. Another consideration is that the data tends to be noisy and inaccurate. Accumulating data over time makes it possible to remove some of the noise and improve accuracy.

To allow for combining data from multiple frames, the observations (point clouds) need to be aligned one way or another. Methods introduced in \autoref{literature.alignment} were available, each with their own trade-offs.

\fixme{write more about this when the best approach is actually chosen}

\section{Mesh generation}

\fixme{Alternative approaches, including the following}

\subsection{Voxel grid}

One possible approach is to use a voxel grid similar to the one Kinect Fusion \citep{} uses. This allows generating meshes representing arbitrary shapes.

To evaluate the approach, a prototype was built on top of Kinfu\footnote{Kinfu is an open source implementation of Kinect Fusion \citep{newcombe2011kinectfusion}. It is included in the Point Cloud Library \citep{PCL}.}. Since the point clouds were already segmented by body part, it was possible to create recordings that only include a single body part. These recordings could then be played back and used as input for Kinfu.

The working hypothesis was that each body part could be treated as a static object, and that Kinfu should do quite well at modeling them. Notably there's little difference between the camera moving (as is the case in Kinect Fusion) and the object moving. If the background is filtered out, the result is similar. This made for the case that running the Kinect Fusion algorithm in parallel to each body part could give reasonably good results.

In practice, Kinfu doesn't work well with isolated limb data. This has multiple causes. \fixme{1) For full-body scanning the whole user must obviously fit in the picture. At the VGA resolution that Kinect uses, this leaves few pixels per limb. 2) The depth resolution of Kinect is about 1--2 centimeters at a distance suitable for full-body scanning. 3) ICP only uses surface features for alignment, not color data. As the individual body parts tend to be quite smooth, this is a problem.}

\subsection{Parametric surfaces}

Another approach is to use the information readily available about the body parts being modeled. Instead of allowing arbitrary shapes, the space of possible shapes can be limited to what the body parts tend to look like.

As the very first prototype, a simple cylinder approximation was used. A cylinder was fitted to the point cloud of a body part, and colored according to its average color. This was good for testing the segmentation and getting a general idea of how accurate the skeleton is. Some corner cases were also found using this prototype.

Obviously, a more human-like model was needed. The "cylinder man" could be usable as an avatar, and is certainly recognizable as human-like. But certainly its shape was nowhere near a real human body. Different simple shapes such as ellipsoids were considered. This would still leave the problem of connecting the body parts. Some kind of a meta-blob solution might be feasible, but in the end this approach started to seem quite cumbersome. And still the accuracy would leave much to be hoped for.

The best parametric model for the purpose would then be something that already assumes the generic human shape, while allowing \fixme{exact} modifications to single body parts. The SCAPE model \citep{anguelov2005scape} used by \citet{weiss2011home} fits the description, but is not suitable for real-time evaluation. A more suitable approach is taken by MakeHuman \citep{makehuman}, which uses a base mesh and has defined parameters that can be used to reshape different parts of the mesh.

MakeHuman is still in development and at the time of writing was undergoing some large changes. It's not designed to be used as a library, either. Interfacing with other software thus seemed nontrivial. After some time tinkering with the MakeHuman implementation, it was decided that the data is important while the software can be replaced. The most important features weren't very difficult to implement.

\fixme{Where should the implementation details of MakeHuman be described?}

\section{Texturing}

\fixme{Write something here once I've done something...}



% after discussion with Tassu, looks like no Results chapter
% \chapter{Results}

\fixme{TODO: assess how well the chosen approach works. Preferably use at least one well-defined quantitative measure.}

\fixme{TODO: go through all the partial implementations, tell what they do and how they fail}

\fixme{TODO: add measurement data; do \bs{analysis} on them}

\chapter{Discussion}
% TODO: reflect on the work done; analyze the choices made etc. as objectively
% as possible

In this chapter, we discuss the work completed towards reconstructing a 3D mesh of the human body and outline suggestions for making a comprehensive implementation. We also analyze the choices made and the operability of different parts of the system. Finally, we conclude by reviewing the major achievements of this thesis, while examining problems in planning and conducting the research.

\fixme{TODO: graphics showing 1) kinect depth 2) point cloud 3) segmenting 4) measurement 5) calculation 6) makehuman}

\fixme{ \\
    TODO: review the best/chosen approach step by step \\
    - kinect data via openni \\
    - user segmentation by NITE \\
    - body part segmentation (nearest neighbor bone) \\
    - automatic measurements on bones and body parts \\
    - ??? \\
    - makehuman measurement plugin \\
    - (blender for animation) \\
}


\section{Future work}

\fixme{TODO: write this properly \\
- find best measurements and their equivalence to parameters \\
- implement real-time fitting of targets \\
- optimize body part segmentation \\
- limbwise kinfu with something else than ICP \\
- texturing \\
- animation (makehuman? blender? something else?) \\
- research Klaus's point cloud accumulation \\
}

\section{Conclusion}

The major accomplishments of this thesis are threefold in regard to human body reconstruction using depth camera data.

Firstly, we conducted a review of the subject matter, including existing systems and related research. In chapter~\ref{literature}, we described the most relevant areas of research in terms of what should be considered when designing a system for 3D reconstruction of the human body. We also encountered and identified the major problems that must be solved in order to create a working system.

Secondly, we suggested different valid approaches in chapter~\ref{approach}. We described each approach in terms of how the problem can be divided into sub-tasks and determined requirements for their implentations. We also analyzed possible problems with each approach and recommended the one we deemed most promising.

Thirdly, we made prototype implementations for many of the tasks described, as described in chapter~\ref{approach}. These implementations helped us further understand the tasks and problems that must be solved. They also provide a useful basis for further work on a complete human body reconstruction system.

\newtopic

Ultimately, this research did not meet its original goal---a complete human body reconstruction system that works in real time. Various causes contributed to this result. Fundamentally, the difficulty of achieving the chosen research goal was seriously underestimated. Our lack of previous knowledge of the subject matter was the primary reason for this. Unfortunately, there was little existing expertise in relevant research topics in the Department of Media Technology, and instruction during the research was lacking. Therefore, it took us long to realize our mistake.

Still, we have made a promising start in the research---the error was mostly in choosing the scope to fit the limited resources of a Master's thesis. Scoping the research appropriately was made difficult by the broadness of the subject. Any given task requires a broad understanding of different research areas, which had to be gained first. We also made the error of focusing on finding a lot of different ideas---while neglecting validation of their feasibility before committing to working on them.

All things considered, we made good groundwork by investigating, planning and partially implementing a system for human body reconstruction, while the research goal was not completely met.

% ...


% Load the bibliographic references
% ------------------------------------------------------------------

% TODO: should all sources be displayed?
\nocite{*}
\printbibliography[heading=bibintoc]


% Appendices go here
% ------------------------------------------------------------------
% If you do not have appendices, comment out the following lines
%\appendix
%\input{appendices.tex}

% End of document!
% ------------------------------------------------------------------
% The LastPage package automatically places a label on the last page.
% That works better than placing a label here manually, because the
% label might not go to the actual last page, if LaTeX needs to place
% floats (that is, figures, tables, and such) to the end of the 
% document.
\end{document}
